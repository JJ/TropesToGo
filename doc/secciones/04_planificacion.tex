\chapter{Planificación}
En este capítulo se especifica la metodología de desarrollo que se ha utilizado para guiar el completo desarrollo del trabajo, indicando la planificación temporal, historias de usuario e hitos que han servido para saber en todo momento el estado del desarrollo, con el objetivo de garantizar la calidad del proyecto y que se cumplan sus objetivos correctamente.

\section{Metodología de desarrollo}
Todo el proyecto, desde su concepción, se ha planteado siguiendo una metodología de desarrollo ágil que siga los principios del manifiesto ágil \cite{agilemanifesto}. El principal objetivo al seguir una metodología ágil es el de asegurar un desarrollo flexible, ágil, centrado en las necesidades de los usuarios a los que se desarrolla el software y en el que exista una tolerancia al cambio.

\begin{itemize}
    \item El desarrollo es iterativo e incremental, en el cual se definen una serie de hitos (o entregables) que conforman un Producto Mínimamente Viable (PMV). Este PMV conforma una parte de la funcionalidad total que se quiere tener al final para cumplir los objetivos especificados, y los tests se asegurarán de que sea completamente funcional, garantizando su calidad.
    \item El desarrollo está organizado por una serie de historias de usuario que definen de forma precisa y adecuada el funcionamiento del software y cómo organizar el desarrollo del mismo, siempre desde el punto de si satisfacen al usuario, que es para quien se desarrolla.
    \item Durante todo el desarrollo se acepta que surjan problemas o nuevos requisitos, que se documentarán adecuadamente y se incorporarán a la planificación en mitad del desarrollo para adaptarse a esos cambios lo antes posible y siempre cumplir las necesidades de los usuarios.
    \item Se busca siempre la calidad del software, usando para ello las mejores prácticas del lenguaje tanto en código como en su documentación, teniendo que estar todo debidamente testeado. Se va a desarrollar una herramienta que en última instancia será utilizada usuarios reales y estos deben obtener el mejor producto posible.
\end{itemize}

Este tipo de desarrollo se adapta bien a un proyecto como este, en el que pueden ir surgiendo necesidades conforme se vaya desarrollando el scraper, puesto que es un problema del cual no se sabe desde el principio todas las necesidades que va a plantear debido a la complejidad de la web de TvTropes y a la de las propias herramientas de scraping. Por tanto, esta metodología permite que los problemas que puedan surgir en medio del desarrollo se aborden a tiempo, documentándolos correctamente y sabiendo en todo momento qué se ha hecho, cuál es el estado del desarrollo del proyecto, qué hacer y cómo hacerlo.

Con el propósito de cumplir correctamente todos los principios ágiles que se han definido se emplean varias herramientas para el seguimiento del desarrollo y que se describen a continuación.

\section{Seguimiento del desarrollo}
\subsection{GitHub}
Al ser un proyecto de software libre el uso de Git y GitHub como software de control de versiones es imprescindible. Esto permite tener un seguimiento cercano de todos los avances que se hacen tanto en el código como en la documentación, que están ambos alojados en el repositorio público\footnote{\url{https://github.com/jlgallego99/TropesToGo}}. Además de esto también facilita el poder recuperar versiones anteriores del software en caso de algún fallo o cambio sustancial, por lo que es ideal en una metodología ágil en la cual existen constantemente cambios.

Sin embargo, lo más importante es que se guía todo el proceso de desarrollo mediante el repositorio de GitHub, teniendo en un mismo lugar todo lo necesario para hacer un correcto seguimiento del estado del trabajo que se tiene en cualquier momento mediante las herramientas que proporciona GitHub. El funcionamiento es el siguiente:
\begin{itemize}
    \item Para llevar el seguimiento de las historias de usuario y tareas asociadas a ellas de la metodología de desarrollo ágil se han creado una serie de issues en el repositorio de GitHub que especifican el trabajo que hay que realizar. Estos issues servirán para saber en todo momento en qué trabajar y siempre están orientados a resolver una historia de usuario, es decir, satisfacer sus necesidades.\\
    Cada vez que, en mitad del desarrollo, se necesite desarrollar algo o solucionar un problema se documenta en un issue que especifica qué se quiere conseguir y entonces se trabaja en cumplirlo.
    \item Los distintos hitos estarán documentados en la sección de milestones, indicando a modo de resumen el objetivo que se pretende alcanzar al tener ese producto funcional. Este es el principal artefacto para conocer el estado del proyecto, ya que cada hito indica una fecha límite para completarlo y una serie de issues necesarios para completarlo por completo. Conforme se van completando las tareas se puede ver cuánto trabajo queda por hacer.
    \item Cada commit es un avance en el código o la documentación, y en el mensaje se hará una pequeña indicación que sirva para saber qué se ha hecho para avanzar en el issue al que hace referencia. De esta manera podemos saber en todo momento las decisiones que se han tomado para resolver una determinada tarea, teniendo un historial con todos los cambios.
    \item Todo el código y documentación que esté en la rama principal del repositorio se entiende que está probado y, por tanto, es completamente funcional. \\
    El código se desarrolla en ramas separadas a la principal, que hacen referencia a cada hito que se quiere alcanzar. Cada vez que se complete una tarea de desarrollo se hará un pull request a la rama principal, y si pasa los flujos de CI configurados (los cuales se explicarán en próximas secciones) entonces se dará la tarea por completada y se tendrá una nueva versión funcional en la rama principal. 
\end{itemize}

Como podemos ver, mediante GitHub se tiene en un mismo sitio todo lo necesario para el desarrollo: código, documentación y seguimiento del desarrollo para saber siempre el trabajo que se tiene que hacer y poder avanzar de una forma más cómoda en alcanzar los objetivos y satisfacer al usuario.

\subsection{Hitos}
Los hitos o milestones representan cada uno de los estados en los que va a ir evolucionando la aplicación, conformando un Producto Mínimamente Viable. Cada PMV se construye encima de los anteriores, por lo que no se avanza en el desarrollo hasta que un hito quede completo en su totalidad. Este se considera como completado una vez se hayan resuelto todas sus tareas asociadas, o issues de GitHub.

Los hitos definidos para el desarrollo son los siguientes:

\subsubsection{M0. Configuración inicial - Objetivos, metodología e infraestructura}
En este milestone se pretende abordar toda la planificación inicial del proyecto, incluyendo los primeros capítulos de la documentación con la descripción, motivación, objetivos, metodología a seguir, historias de usuario y estado del arte.

El objetivo es tener una configuración inicial del repositorio, configurando el comprobador ortográfico para la documentación, eligiendo el lenguaje de programación que se va a usar, el task runner, los sistemas de CI y preparando todos los issues, historias de usuario y milestones en el propio repositorio.

Esta iteración se desarrolla entre el \textbf{06/02/23} y el \textbf{07/03/23}, teniendo una duración completa de un mes.
\subsubsection{M1. Extracción básica de tropos de películas}
El objetivo principal de este hito es tener una versión básica y funcional del scraper que extrajese información básica sobre las películas únicamente. Se busca desarrollar las clases y módulos necesarios que contengan la funcionalidad necesaria para extraer todas las películas existentes en TvTropes y sus tropos asociados, integrar el código en un framework de test y probar todas las funcionalidades desarrolladas.

\subsubsection{M2. Extracción de tropos de todos los tipos de medios audiovisuales}
Una vez habiendo extraído la información sobre tropos de películas, este hito construye sobre el anterior ampliando el scraper para que ahora saque información de todos los tipos de medios audiovisuales existentes en TvTropes, testeando todas las funcionalidades desarrolladas.

\subsubsection{M3. Integración con IMDB y extracción de metadatos}
Una vez teniendo un scraper de TvTropes completamente funcional, se amplía el producto para que todos los medios audiovisuales extraídos se puedan relacionar con su ficha correspondiente en IMDB y otras diversas fuentes de datos sobre obras audiovisuales, añadiendo a la información los metadatos de cada medio, testeando todas las funcionalidades desarrolladas.

\subsection{Desarrollo dirigido por pruebas}

\subsection{Integración continua}

\section{Usuarios o partes interesadas}
Antes de definir las historias de usuario, se han analizado los usuarios o partes interesadas en el proyecto, para enfocar el proyecto en resolver sus problemas. Son los siguientes:
\begin{itemize}
    \item \textbf{Tribunal}: el tribunal es el encargado de valorar este trabajo de fin de máster y, por tanto, busca que tenga una memoria bien redactada con todo el proceso de estudio, análisis y desarrollo del proyecto.
    \item \textbf{Investigador}: el investigador, o analista de datos, necesita tener la información estructurada de los tropos para construir modelos que le permitan hacer inferencias sobre las relaciones que tienen los tropos entre sí y llegar a conclusiones que sirvan para su investigación.
\end{itemize}

\section{Historias de usuario}
Sabiendo el problema que se quiere resolver, y las partes interesadas, se pueden definir las historias de usuario que guiarán todo el desarrollo de principio a fin, guiando las decisiones que se irán tomando para llegar a cumplir los objetivos.

\subsection{[HU01] Investigador - Obtener información estructurada sobre todos los tropos}
Como investigador necesito poder obtener información estructurada sobre todos los tropos existentes de cualquier tipo de medio audiovisual, en un formato estandarizado y con una estructura definida para que sean de fácil acceso tanto a humanos como programas para su análisis.

\subsection{[HU02] Investigador - Obtener un subconjunto de la información}
Como investigador necesito poder elegir qué parte de la información sobre tropos quiero obtener, ya sea de un medio audiovisual concreto, todo un género, etc.

\subsection{[HU03] Investigador - Relacionar medios audiovisuales con su ficha en IMDB}
Como investigador necesito obtener información de metadatos sobre cualquier medio audiovisual, para lo cual necesito que los medios audiovisuales de los cuales se han sacado sus tropos se puedan relacionar con su ficha en IMDB.

\subsection{[HU04] Investigador - Información actualizada}
Como investigador quiero que la información de tropos que obtenga esté lo más actualizada posible, sepa cuándo se ha extraído y pueda actualizarla cuando quiera para traer la nueva información que pueda existir.

\section{Temporización}
Una vez vistos los hitos que se esperan alcanzar en el desarrollo, los cuales conformarían cada uno un sprint con una fecha de inicio y de fin, podemos representar la temporización de todo el desarrollo en un diagrama de Gantt.

A lo largo de todo el desarrollo se ha ido manteniendo contacto con el tutor tanto en tutorías presenciales como mediante mensajes por Telegram y GitHub para transmitir y resolver todas las dudas que han ido surgiendo, tanto del código como del propio documento.